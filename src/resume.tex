% Ankit Sachdeva 2022
% resume.tex v3.0
% Please read /src/README.md and LICENSE

% PREABMLE
\documentclass{resume}
\name{Ankit Sachdeva}
\usepackage{fontspec}
\setmainfont{PT Sans}
\usepackage[left=0.3in, top=0.3in, right=0.3in, bottom=0.3in]{geometry}
\usepackage[hidelinks]{hyperref}
\usepackage[normalem]{ulem}
\RequireXeTeX
\address{
\href{mailto:ankit@ucsc.edu}{{ankit@ucsc.edu}} \\
%\href{https://queer.ucsc.edu/education-resources/pronouns-matter.html}{{he, him, his}} \\
\href{https://ankitsachdeva.com/}{{ankitsachdeva.com}} \\
\href{}{{Cupertino,  CA}} \\
\href{https://github.com/ankitsxchdeva}{{github.com/ankitsxchdeva}} \\
\href{https://www.linkedin.com/in/ankitsxchdeva/}{{in/ankitsxchdeva}}
}


\begin{document}

\begin{rSection}{\large Education}

{\bf Georgia Institute of Technology} \hfill {\bf{June 2024 - Current}}
\\ Masters of Science (M.S.) in Computer Science\hfill {Online \& Part-Time}
\\ \textbf{Current Coursework:} Database System Concepts and Design, Software Development Process


{\bf University of California, Santa Cruz} \hfill {\bf{September 2020 - June 2024}}
\\ Bachelor of Science (B.S.) in Computer Science and Engineering\hfill {GPA: 3.8/4.0}
\\ \textbf{Relevant Coursework:} Operating Systems, Data Structures and Algorithms, Computer Architecture, Computer Networking, Computer Systems Design, Distributed Systems, Assembly Language, Embedded Systems, Artificial Intelligence

\end{rSection}

\begin{rSection}{\large Experience}

\begin{rSubsection}{PayPal}{\bf{San Jose, CA}}{Software Engineer Intern}{June 2022 - September 2022}
\item Investigated the potential of moving the settlement team's file exchange processes to a blockchain based stack
\item Designed and implemented a proof-of-concept with solidity to collect metrics including throughput, reliability and robustness
\end{rSubsection}

\begin{rSubsection}{Baskin School of Engineering at UCSC}{\bf{Santa Cruz, CA}}{Group Tutor}{March 2022 - Current}
\item Group tutor for Computer Systems and C Programming, a course focusing on computer systems and algorithm design
\item Duties include aiding students with debugging code, grading exams, and maintaining course infrastructure such as GitLab CI/CD
\end{rSubsection}

\begin{rSubsection}{Tech4Good Lab at UCSC}{\bf{Santa Cruz, CA}}{Undergraduate Research Assistant}{January 2022 - March 2022}
\item Modified the \href{https://github.com/salesforce/ai-economist}{Salesforce AI Economist tax model} to analyze scaling of apprenticeship learning under Professor David Lee
\item Added actions and other variables to new and existing agents in order to model different styles of apprenticeship programs
\end{rSubsection}

\begin{rSubsection}{Fox Factory}{\bf{Scotts Valley, CA}}{Embedded Software Engineer Intern}{September 2021 - January 2022}
\item Performed QA related tasks such as creating test plans, conducting regression testing, and overseeing environmental testing
\item Developed firmware for the \href{https://www.pinkbike.com/news/fox-updates-live-valve-electonic-suspension-for-2022.html}{Live Valve project's} embedded systems in C utilizing the Nordic nRF52 SDK and SoC
\item Designed Python tools to be used in EOL testers to perform QA related tasks, verify hardware and firmware functionality
\end{rSubsection}
\end{rSection}

\begin{rSection}{\large Projects}

\begin{rSubsection}{Pintos Operating System}{\href{http://ankitsachdeva.com/pintos/}{\bf{{ankitsachdeva.com/pintos}}}}{}{}
\item Modified the Pintos educational operating system to support priority-based thread scheduling and priority donation between threads
\item Added support for a more efficient version of the "timer sleep" system call that improves performance by removing busy-waiting
\end{rSubsection}

\begin{rSubsection}{Huffman Compression Algorithm}{\href{https://ankitsachdeva.com/huffman/}{\bf{{ankitsachdeva.com/huffman}}}}{}{}
\item Implemented the lossless Huffman Compression algorithm in C with low level system calls for I/O reads and writes
\item Created and utilized fundamental data structures including nodes, queues and stacks and performed bit-wise operations
\end{rSubsection}

\begin{rSubsection}{Unmasked Android Application}{\href{https://ankitsachdeva.com/unmasked}{\bf{{ankitsachdeva.com/unmasked}}}}{}{}
\item Designed and wrote an Android application to scan cosmetic items and highlight potentially harmful or allergic ingredients
\item Utilized Firebase and Google OCR API with image enhancement, written with a mix of Kotlin and Java in Android Studio
\end{rSubsection}

\begin{rSubsection}{Photography Portfolio}{\href{https://ankitsachdeva.com/photography}{\bf{{ankitsachdeva.com/photography}}}}{}{}
\item Designed and wrote an Android application to scan cosmetic items and highlight potentially harmful or allergic ingredients
\item Utilized Firebase and Google OCR API with image enhancement, written with a mix of Kotlin and Java in Android Studio
\end{rSubsection}
\end{rSection}

\begin{rSection}{\large Skills}

\begin{rSubsection}{Programming Languages}{}{}{}
\item Go, Python, Bash, Java, C/C{}\verb!++!, MIPS Assembly, RISC-V Assembly, SQL, HTML/CSS, JavaScript, Kotlin, Swift 
\end{rSubsection}

\begin{rSubsection}{Technologies}{}{}{}
\item Git, SVN, Flask, Node.js, Express, React, MongoDB, NumPy, Pandas, Matplotlib, LaTeX, Docker/Podman, AWS
\end{rSubsection}
\end{rSection}

\end{document}
